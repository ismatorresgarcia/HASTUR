% !TeX program = lualatex
%% La primera línea es un mapping entre la directiva TeX y el motor de compilación que utiliza VimTeX
%%% clase de base (KOMA-Script)
\documentclass{scrartcl} %% clase artículo KOMA-Script para el documento base
\KOMAoptions{     
  paper = a4,                         %% tamaño del papel
  DIV = 12,                           %% geometría
  BCOR = 0mm,                        %% geomtría
  fontsize = 11pt,                    %% tamaño de la fuente
  twoside = false,                    %% no recto-verso
  headings = normal,                  %% tamaño de letra para los títulos: small, normal, big
  parskip = half,                     %% sangría y espaciado de párrafos
  %headsepline = true,                  %% una linea separa la cabecera del texto
  %listof = totoc,                      %% añadir a los contenidos la lista de tablas y figuras
  %bibliography = toc,                  %% añadir a los contenidos una entrada para la bibliografia
  %draft                              %% para versiones de revisión, muestra las líneas mal formateadas
}
%%% paquetes básicos
%% fórmulas, símbolos y teoremas matemáticos. 'unicode-math' sobrescribirá las definiciones necesarias después
\usepackage{amsmath}              %% fórmulas y entornos matemáticos. Cargar antes de 'unicode-math' por recomendación de este  
\usepackage{amssymb}              %% fuentes y símbolos matemáticos (carga 'amsfonts' también)

\usepackage{unicode-math}         %% gestión de matemáticas. Los paquetes de la AMS son incompatibles, así que ponlos antes para que sobrescriba lo que considere necesario. También carga 'fontspec' para gestionar las fuentes de texto con XeLaTeX o LuaLaTeX
\unimathsetup{                        %% las opciones del paquete 'unicode-math'
  math-style = TeX, 
  bold-style = TeX,
}
\setmainfont{minionpro-regular}[       %% Fuente comercial Minion Pro de Adobe (desarrollador Robert Slimbach) para el texto en Serif
  Extension      = .otf,
  Path = /Users/ytoga/Documents/fuentes-texto/minion-pro/OTF-Fonts/,
  BoldFont       = minionpro-bold,
  ItalicFont     = minionpro-it,
  BoldItalicFont = minionpro-boldit
]
\setmathfont{MinionMath-Regular}[      %% Fuente comercial Minion Math de Johannes Küsner para la matemática
  Extension = .otf,
  Path = /Users/ytoga/Documents/fuentes-texto/minion-math/fonts-MinionMath/,
  Scale = 1,
  Script = Math,
  SizeFeatures = {
    {Size = -6, Font = MinionMath-Tiny,
    Style = MathScriptScript},
    {Size = 6-8.4, Font = MinionMath-Capt,
    Style = MathScript},
    {Size = 8.4-, Font = MinionMath-Regular}
  }
]
\setmathfont{MinionMath-Bold}[        %% Fuente comercial Minion Math de Johannes Küsner para la matemática
  Extension = .otf,
  Path = /Users/ytoga/Documents/fuentes-texto/minion-math/fonts-MinionMath/,
  version = bold
]
\setmathfont{XITSMath-Regular}[       %% Fuente XITSMath para la matemática (Minion Math no tiene símbolos para gótica ni caligráfica)
  Extension = .otf,
  Path = /Users/ytoga/Documents/fuentes-texto/xits/,
  BoldFont = XITSMath-Bold,
  range = {\symcal,\symfrak,\symbfcal,\symbffrak},
]

%%% microtipografía de las fuentes con 'microtype'. Con LuaLaTeX tiene más funcionalidades (expansión y protrusión) que XeLaTeX (solo protrusión)
\usepackage{microtype}            %% microtipografía (detalles finos entre palabras, en los márgenes, etc.)

%%% gestión de los idiomas. Las opciones de 'polyglossia' las cargo más adelante por recomendación del manual
\usepackage{polyglossia}
\setdefaultlanguage[ %%% configuración de los idiomas con 'polyglossia'
  variant = spanish,
  spanishoperators = all,
]{spanish}
\setotherlanguage[variant=british]{english}
\usepackage[autostyle]{csquotes}        %% entorno para comillas tipográficas (y más cosas) 

%%% matemáticas, física y química
\usepackage{siunitx}                    %% unidades del sistema internacional
\sisetup{                                   %% opciones de las unidades del paquete
  output-decimal-marker = {,},
  exponent-product = \times,
  range-phrase = \textup{--},
  range-exponents = combine-bracket,
  range-units = repeat,
  uncertainty-mode = separate,
  list-pair-separator = { | },
  per-mode = symbol
}

%%% tablas y figuras
\usepackage{graphicx}             %% imágenes y figuras
\graphicspath{{Figuras/}}             %% ruta de las figuras
\usepackage{booktabs}             %% tablas profesionales

\usepackage{lipsum} %% párrafos de relleno

%%% configuración de las páginas en KOMA-Script
\usepackage[
    automark,                              
    autooneside=false                       
]{scrlayer-scrpage}  
\pagestyle{scrheadings}

%%% colores del documento
\usepackage{xcolor}
\definecolor{miazul}{RGB}{0,30,155}
\definecolor{mirojo}{RGB}{153,0,0}
\definecolor{mirojoo}{RGB}{200,100,80}
\definecolor{miverde}{RGB}{0,153,0}
\definecolor{mimorado}{RGB}{140,45,165}
\definecolor{minaranja}{RGB}{250,150,5}

%%% PDF
\usepackage{bookmark}                   %% gestión moderna de los marcadores del paquete 'hyperref'
\usepackage{hyperref}                   %% pdf con hipervínculos, referencias cruzadas, etc.
\hypersetup{
  %backref=true,                          %% añadir hipervínculos (por defecto)
  %pagebackref=true,                      %% en la bibliografía (por defecto)
  %hyperindex=true,                       %% en el índice (por defecto)
  %bookmarks=true,                        %% marcadores de Acrobat (por defecto)
  breaklinks = true,                        %% romper línea si es demasiado largo el hipervínculo
  colorlinks = true,                        %% color de hipervínculos
  urlcolor = minaranja,                     %% color de hipervínculos de direcciones web
  citecolor = miverde,                      %% color de citas bibliográficas internas
  linkcolor = miazul,                       %% color de vínculos internos
  filecolor = mirojo,                       %% color de urls que abren archivos
  menucolor = mirojoo,                      %% color de las pestañas en Adobe
  anchorcolor = black,                      %% color del texto anclado cuando pasas por encima
  bookmarksopen = false,                    %% las pestañas de los marcadores del pdf aparecen todas desplegadas
  %linktocpage=false,                     %% hipervínculos en el número de página en lugar del texto del TOC
%% metacampos del pdf :                     %% ATENCIÓN: hay que completarlos 
  pdftitle    = {Modelo Reducido},
  pdfauthor   = {Ismael Torres García},
  pdfsubject  = {Filamentación},
  %pdfkeywords = {SXRL, Kriptón, XUV, HOH, HHG, Dagon},
}

%%% configuración de la bibliografía con 'biblatex'. Biblatex carga 'etoolbox' entre otros paquetes
\usepackage[                            %% gestion de la bibliografía con biblatex
  backend = biber,                          %% compilador por defecto con biblatex (recomendado)
  maxnames = 20,                            %% número max. de nombres de autores
  minnames = 1,                             %% número min. de nombres de autores
  sorting = nyt,                            %% ordenar por nombre, título y año 
  bibstyle = numeric,                       %% estilo de bibliografía numérica (pon numeric, en draft mientras edito)  
  citestyle = numeric-comp,                 %% estilo de las citas numérico tipo [1-3] (pon numeric-comp, en draft mientras edito)
  backref = false                           %% sacar las páginas de las citas en la bibliografía (cf. p.)
]{biblatex}                   
\addbibresource{Zotero.bib}                 %% archivo bibliográfico que estoy utilizando

%%% comandos para configurar tamaños y fuentes de los distintos niveles
\addtokomafont{disposition}{\normalfont\bfseries}

%\setkomafont{chapter}{\sffamily\fontsize{25pt}{30pt}\selectfont\color{miazul}}
%\setkomafont{chapterprefix}{\sffamily\fontsize{11pt}{13pt}\selectfont\color{miazul}}
%\newkomafont{chapternumber}{\fontsize{50pt}{60pt}\selectfont\color{miazul}}
%
%\addtokomafont{section}{\sffamily\fontsize{17pt}{21pt}\selectfont\color{miazul}}
%\addtokomafont{subsection}{\sffamily\fontsize{14pt}{17pt}\selectfont\color{miazul}}
%\addtokomafont{subsubsection}{\sffamily\fontsize{11pt}{13pt}\selectfont\color{miazul}}
%\addtokomafont{paragraph}{\normalfont\sffamily\itshape\fontsize{12pt}{14pt}\selectfont\color{miazul}}
%\addtokomafont{subparagraph}{\normalfont\sffamily\itshape\fontsize{12pt}{14pt}\selectfont\color{miazul}}

%%% Definición de comandos útiles para escribir con los snippets
%% vectores de la base
\newcommand*{\unitvector}{\symbf{e}}
\newcommand*{\uvec}{\hat{e}}
%% derivadas totales
\newcommand*{\dvOne}[1]{\frac{\operatorname{d}}{\operatorname{d}\!#1}}
\newcommand*{\dv}[2]{\frac{\operatorname{d}\!#1}{\operatorname{d}\!#2}}
\newcommand*{\dvN}[3]{\frac{\operatorname{d^{#3}}\!#1}{\operatorname{d}\!{#2}^{#3}}}
%% derivadas parciales
\newcommand*{\pdvOne}[1]{\frac{\partial}{\partial#1}#1}
\newcommand*{\pdv}[2]{\frac{\partial#1}{\partial#2}}
\newcommand*{\pdvN}[3]{\frac{\partial^{#3}#1}{\partial{#2}^{#3}}}
%% diferenciales de primer orden
\newcommand*{\diff}{\operatorname{d}\!}
\newcommand*{\pdiff}{\partial}
%% nombres de conceptos
\newcommand*{\RE}{\operatorname{Re}}
\newcommand*{\IM}{\operatorname{Im}}
%% conjuntos de números
\newcommand*{\iu}{\symrm{i}}
\newcommand*{\eu}{\symrm{e}}
%% operadores matemáticos
\newcommand*{\grad}{\nabla}
\newcommand*{\curl}{{\nabla}\times}
\newcommand*{\diver}{\nabla\cdot}
\newcommand*{\laplacian}{\Delta}
%% combinatoria, probabilidad y estadística
\newcommand*{\bnn}[2]{\binom{#1}{#2}}
%% delimitadores importantes
\newcommand*{\abs}[1]{\left|#1\right|}
\newcommand*{\bra}[1]{\langle#1|}
\newcommand*{\ket}[1]{|#1\rangle}
\newcommand*{\xval}[1]{\langle#1\rangle}
\newcommand*{\Xval}[3]{\left\langle#1\middle|#2\middle|#3\right\rangle}
\newcommand*{\braket}[2]{\left\langle#1\middle|#2\right\rangle}

%%% primera página
%\titlehead{}         
%\subject{}
\title{Filamentación. Procesos físicos y métodos numéricos}
\author{Ismael Torres García}

\begin{document}
\maketitle

%%% resumen
\begin{abstract}
Obtención de las ecuaciones que describen la filamentación de pulsos láser ultraintensos (intensidades entre $\sim \qty{e13}{W/cm^2}$ y $\sim \qty{e14}{W/cm^2}$) y ultracortos (duraciones entre $\sim \qty{10}{fs}$ hasta $\sim \qty{1}{ps}$) con gases, líquidos o sólidos. Solución analítica de las ecuaciones que describen la interacción lineal del pulso. Relación entre la evolución espacial de la intensidad del pulso y la acumulación de fase debido a los fenómenos de difracción y dispersión. Presentación de los métodos numéricos empleados habitualmente para resolver las ecuaciones de propagación para la envolvente del pulso.
\end{abstract}

%%% índice general
\tableofcontents

%%% secciones
\section{Ecuaciones de propagación unidireccional del pulso}\label{sec:1}
Los modelos de filamentación típicos realizan una descomposición del campo eléctrico $\symbf{E}(x,y,z,t) \in \symbf{R}^{3}$ en cada punto del espacio y del tiempo en una envolvente $\symcal{E}(x,y,z,t) \in \symbf{C}$ y una aplicación exponencial definida como $c_{\omega}(x,y,z,t) = \exp({i(k_{x}x + k_{y}y + k_{z}z - \omega t)}) \in \symbf{C}$, con $k_{x}$, $k_{y}$ y $k_{z}$ las componentes del vector número de onda según las direcciones ortogonales del espacio $\symbf{e}_{x}$, $\symbf{e}_{y}$, $\symbf{e}_{z}$, y con $\omega$ la frecuencia central del paquete de ondas. A partir de este momento, cualquier vector aparecerá escrito en negrita mediante una lista de sus componentes en la base canónica de $\symbf{R}^{3}$, definiéndose (aparecerán seguidos por el símbolo $\triangleq$) la posición $\symbf{r} \triangleq (x,y,z)$ y el número de onda $\symbf{k} \triangleq (k_{x},k_{y},k_{z})$. Mediante esta descomposición, un campo eléctrico polarizado linealmente en dirección $\symbf{e}_{x}$ propagándose en dirección $\symbf{e}_{z}$ quedaría expresado como 
\begin{equation}\label{eq:1}
  \symbf{E}(\symbf{r},t) = \RE (\symcal{E}(\symbf{r},t)\exp(i(kz - \omega t)))\symbf{e}_{x},
\end{equation}
siendo la fase $\phi(\symbf{r},t) = \symbf{k}\cdot \symbf{r} - \omega t$ dependiente de la posición y del tiempo, y $k = \abs{\symbf{k}} = 2 \pi / \lambda$ el módulo para el número de onda. 

Una polarización lineal del campo asumirá dos cosas, una es adicional y otra es obvia: Primero, la respuesta del medio también sigue la misma dirección de vibración del campo eléctrico, haciendo que el término $\grad (\diver \symbf{E})$ que aparecerá en la ecuación de ondas sea despreciable. Dicho de otro modo, los campos buscados serán solenoidales o transversales. Mientras que los pulsos láseres no estén fuertemente enfocados en un punto, la componente del campo según la coordenada $z$ es muy pequeña y esta condición es válida. Segundo, la dirección de vibración es lineal, es decir, puede escogerse un sistema de coordenadas donde las ecuaciones quedan reducidas a estudiar campos escalares en lugar de vectoriales.

Campos eléctricos descritos por la ecuación \eqref{eq:1} es ventajoso debido a la existencia de diferentes escalas de observación para las tres direcciones del espacio durante la filamentación, es decir, permite aprovechar la aparición de una dirección preferencial según la coordenada $z$ ---que representará la propagación--- donde las variaciones espaciales de la envolvente serán pequeñas comparadas con la longitud de onda correspondiente a esa dirección. Esta hipótesis adicional es conocida normalmente como \textit{Slowly Varying Envelope Aproximation} (SVEA), y también será muy importante para obtener las ecuaciones unidireccionales de propagación. 

\subsection{Modelo simple}\label{sec:11}
Hechas las aclaraciones, para comenzar la deducción de las ecuaciones de propagación, partiremos de las ecuaciones de Maxwell asociadas a la ley de Faraday \eqref{eq:2} y la ley de Ampère \eqref{eq:3} en un medio dieléctrico, homogéneo, isótropo y no magnético. Las ecuaciones interesantes para obtener una ecuación de ondas son
\begin{align}
  \curl \symbf{E} + \pdv{\symbf{B}}{t} &= 0, \label{eq:2} \\
  \curl \symbf{H} - \pdv{\symbf{D}}{t} &= \symbf{J}_{f} \label{eq:3},
\end{align}
con $\symbf{B} = \mu_{0} \symbf{H}$ el campo magnético en el medio, $\symbf{D} = \epsilon_{0}\symbf{E} + \symbf{P}$ la corriente de desplazamiento y $\symbf{P} = \symbf{P}_{\mathrm{L}} + \symbf{P}_{\mathrm{NL}}$ la polarización inducida en el medio. La polarización puede separarse en una respuesta lineal con el campo eléctrico $\symbf{P_{\mathrm{L}}} = \epsilon_{0} \chi_{e,1}\symbf{E}$ ---siendo $\chi_{e,1}$ la susceptibilidad eléctrica de primer orden del medio--- y otra respuesta no lineal $\symbf{P_{\mathrm{NL}}}$ que detallaremos más adelante.

Estas ecuaciones \enquote{constitutivas} incorporan la respuesta del medio al campo eléctrico del pulso, de acuerdo con las condiciones establecidas de homogeneidad (propiedades como las susceptibilidades eléctricas no cambian con la posición), isótropo (tampoco cambian con la dirección de observación) y no magnético (en resumen, la permabilidad magnética del medio es una constante igual a la del vacío $\mu \approx \mu_{0}$).

En resumen, para cualquier punto del espacio y en cualquier dirección las susceptibilidades eléctricas son constantes de proporcionalidad que relacionan la polarización lineal y no lineal inducida en el medio por el campo eléctrico, mientras que el magnetismo del medio es ignorado, como si los átomos no respondieran al campo magnético.

Tomando rotacionales en las ecuaciones \eqref{eq:2} y \eqref{eq:3}, e incorporando las ecuaciones constitutivas para la respuesta del material, resulta que campo eléctrico está sujeto a la siguente ecuación
\begin{equation}\label{eq:4}
  \laplacian \symbf{E} - \grad (\diver \symbf{E}) - \mu_{0}\left(\pdv{\symbf{J}_{f}}{t} + \epsilon_{0}\pdvN{\symbf{E}}{t}{2} + \epsilon_{0} \chi_{e,1}\pdvN{\symbf{E}}{t}{2} + \pdvN{\symbf{P_{\mathrm{NL}}}}{t}{2}\right) = \symbf{0},
\end{equation}
recordando que $\curl (\curl \symbf{E}) = \grad (\diver \symbf{E}) - \laplacian \symbf{E}$ mientras las componentes del campo tenga segundas derivadas parciales continuas.

Reconociendo $1 + \chi_{e,1} = n^{2}_{0}$ como el índice de refracción lineal del medio, introduciendo las hipótesis sobre la polarización lineal del campo mencionadas anteriormente y reagrupando los términos tenemos la ecuación de ondas
\begin{equation}\label{eq:5}
  \laplacian E - \frac{n^{2}_{0}}{c^{2}}\pdvN{E}{t}{2} = \mu_{0} \left(\pdv{J_{f}}{t} + \pdvN{P_{\mathrm{NL}}}{t}{2}\right)
\end{equation}
para la única componente del campo según la dirección en la que está polarizado. La constante $c$ es la velocidad de propagación de la luz en el vacío.

A continuación, es necesario concretar la respuesta del material. Primero, vamos a suponer que los iones resultantes de la interacción con el pulso son suficientemente pesados como para considerar sus velocidades medias despreciables en comparación con las velocidades $v_{e}$ de los electrones libres, de manera que $J_{f} \approx -e \rho v_{e}$, siendo $e$ la carga eléctrica del electrón en valor absoluto. La dinámica de los electrones viene dada entonces por
\begin{equation}\label{eq:6}
  \pdv{J_{f}}{t} = -e \rho \pdv{v_{e}}{t} = -e \rho \frac{F_{e}}{m_{e}} = \frac{e^{2}}{m_{e}} \rho E,
\end{equation}
con $m_{e}$ la masa del electrón y $F_{e}$ la fuerza de Lorentz para el campo eléctrico, que sabemos es opuesta a la dirección del campo para los electrones. 

La ecuación \eqref{eq:6} es una simplificación importante que no tiene en cuenta la fuerte interacción no lineal (y cuántica) del pulso con los propios electrones generados, pero servirá para conseguir el modelo sencillo buscado. La próxima sección \S\ref{sec:11} incluirá nuevos términos que completarán el comportamiento de los electrones en estas condiciones.

En segundo lugar, normalmente se escoge como material un medio con \enquote{simetría central} de tercer orden cuya respuesta no lineal es cúbica con el campo eléctrico, de forma que $P_{\mathrm{NL}} = \epsilon_{0} \chi_{e,3} E^{3}$. Esta respuesta es la habitual del llamado efecto Kerr óptico no lineal. La susceptibilidad eléctrica de tercer orden $\chi_{e,3}$ para este tipo de medios es $\chi_{e,3} = 4 n_{2}n_{0}^{2}\epsilon_{0} c / 3$, con $n_{2}$ el índice de refracción no lineal del medio\cite{Couairon2007}. Este término del índice de refracción está medido en unidades de inverso de intensidad, de tal forma que el índice de refracción resultante $n(\symbf{r},t) = n_{0} + n_{2}I(\symbf{r},t)$ es adimensional, definiendo
\begin{equation}\label{eq:7}
  I(\symbf{r},t) = \frac{1}{2} \epsilon_{0} c n_{0} \abs{\symcal{E}(\symbf{r},t)}^{2}
\end{equation} 
como la intensidad del pulso láser.

Finalmente, queda introducir la ecuación \eqref{eq:6} junto con la polarización no lineal $P_{\mathrm{NL}}$ en la ecuación de ondas \eqref{eq:5} y desarrollar cada uno de los sumandos en ambos lados de la ecuación. Para conseguir la ecuación de propagación de la envolvente, sustituimos el campo eléctrico propuesto en la ecuación \eqref{eq:1} expresado como 
\begin{equation}\label{eq:8}
  E(\symbf{r},t) = \frac{1}{2}(\symcal{E}(\symbf{r},t)\exp(i(kz - \omega t)) + \mathrm{c.c})),
\end{equation}
donde $\mathrm{c.c}$ representa el complejo conjugado del primer sumando, esto es, $\symcal{E}^{*}(\symbf{r},t)\exp(-i(kz - \omega t))$. 

A partir de aquí entra en juego la aproximación de la envolvente lentamente variable (SVEA) \cite{Milonni2010} presentada en el inicio de la sección \S\ref{sec:1}. Sencillamente consiste en asumir variaciones espaciales y temporales lentas de la envolvente en un ciclo óptico. Dependiendo de las circunstancias, interesa aplicar esta condición de manera más restrictiva o menos. En nuestro caso, los términos con derivadas de segundo orden en la coordenada espacial $z$ serán despreciables respecto a las de primer order, mientras que las derivadas de segundo orden en el tiempo también serán despreciables respecto al valor de la envolvente en un periodo de oscilación. Por tanto, los módulos de estas derivadas cumplen que
\begin{equation}\label{eq:9}
  \abs{\pdvN{\symcal{E}}{z}{2}} \ll k\abs{\pdv{\symcal{E}}{z}}, \quad \abs{\pdvN{\symcal{E}}{t}{2}} \ll \omega\abs{\pdv{\symcal{E}}{t}},
\end{equation}
de acuerdo con la SVEA.

Otra simplificación importante tiene que mencionarse. El desarrollo de los términos de la ecuación \eqref{eq:6} es bastante tedioso, principalmente el término del efecto Kerr no lineal. Este efecto también es responsable de la generación de armónicos de tercer orden, produciendo términos que van acompañados de aplicaciones tipo $c_{3 \omega}(z,t)=\exp(\pm i(3kz - 3 \omega t))$. Estos modos de vibración correspondientes con el tercer armónico vamos a despreciarlos también, ya que nos interesa estudiar qué ocurre exclusivamente con el armónico fundamental del láser.

Después de esta última incorporación a la lista de hipótesis, la ecuación final resultante es
\begin{equation}\label{eq:10}
  \pdvN{\symcal{E}}{x}{2} + \pdvN{\symcal{E}}{y}{2} + 2ik \pdv{\symcal{E}}{z} + 2i \omega\frac{n^{2}_{0}}{c^{2}} \pdv{\symcal{E}}{t} - k^{2}\symcal{E} + \frac{n^{2}_{0}}{c^{2}} \omega^{2}\symcal{E} = \frac{e^{2}}{\epsilon_{0}m_{e}c^{2}}\symcal{E} - \frac{4}{3}\frac{\chi_{e,3}}{c^{2}} \omega^{2}\abs{\symcal{E}}^{2}\symcal{E}.
\end{equation}

Vamos a introducir la frecuencia de las oscilaciones del plasma $\omega^{2}_{pe} = \rho e^{2} / \epsilon_{0}m_{e}$ y la densidad crítica del plasma $\rho_{c} = \epsilon_{0}m_{e} \omega^{2} / e^{2}$. Mientras que la frecuencia del pulso $\omega \gg \omega_{pe}$ ---es equivalente a que la densidad de electrones $\rho \ll \rho_{c}$---, entonces puede utilizarse como relación de dispersión $\omega^{2} = \omega^{2}_{pe} + k^{2}c^{2} \approx k^{2}c^{2}$. Conociendo la relación de dispersión, el índice de refracción lineal del medio es $n_{0} = ck/\omega$ ---la velocidad de fase y de grupo son iguales porque hemos despreciado la frecuencia del plasma---, obteniéndose de la ecuación \eqref{eq:10} que
\begin{equation}\label{eq:11}
  \pdv{\symcal{E}}{z}  + \frac{n_{0}}{c} \pdv{\symcal{E}}{t} = i \frac{1}{2k} \laplacian_{\perp}\symcal{E} + i \frac{1}{2} \epsilon_{0}cn_{2}k_{0}n_{0} \abs{\symcal{E}}^{2}\symcal{E} - i \frac{k_{0}}{2n_{0}}\frac{\rho}{\rho_{c}} \symcal{E},
\end{equation}
con $\laplacian_{\perp} = \partial^{2}/\partial x^{2} + \partial^{2}/\partial y^{2}$ el \enquote{operador} laplaciano transversal y $k_{0} = k/n_{0}$ el número de onda para el campo eléctrico en el vacío.

El segundo término puede condensarse todavía más midiendo la intensidad y el cuadrado del campo en las mismas unidades, es decir, haciendo que $I = \abs{\symcal{E}}^{2}$. En la definición vista en la ecuación \eqref{eq:7} aparece un factor $\epsilon_{0}cn_{0}/2$ porque está en las unidades del Sistema Internacional, pero adimensionalizándolo y haciendo que sea igual a la unidad entonces
\begin{equation}\label{eq:12}
  \pdv{\symcal{E}}{z} + \frac{n_{0}}{c} \pdv{\symcal{E}}{t} = i \frac{1}{2k} \laplacian_{\perp}\symcal{E} + i k_{0}n_{2}\abs{\symcal{E}}^{2}\symcal{E} - i \frac{k_{0}}{2n_{0}}\frac{\rho}{\rho_{c}} \symcal{E}.
\end{equation}

Normalmente, una práctica muy extendida en filamentación de pulsos láser \cite{Couairon2007, Couairon2011} consiste en estudiar la propagación en un sistema de referencia que viaje a la velocidad del pulso $v_{g} = c/n_{0}$. Mediante el cambio de variable descrito por las ecuaciones $\tau = t - z/v_{g}$ ---la nueva \enquote{variable} $\tau$ suele llamarse tiempo reducido--- y $\zeta = z$ el lado derecho de la ecuación \eqref{eq:12} correspondiente con la advección termina reduciéndose a
\begin{equation}\label{eq:13}
  \pdv{\symcal{E}}{\zeta} = i \frac{1}{2k} \laplacian_{\perp}\symcal{E} + i k_{0}n_{2}\abs{\symcal{E}}^{2}\symcal{E} - i \frac{k_{0}}{2n_{0}}\frac{\rho}{\rho_{c}} \symcal{E}.
\end{equation}

Esta ecuación de propagación unidireccional tiene que ir acompañada de una segunda ecuación que describa la evolución de la densidad electrónica $\rho$ en el tiempo reducido. El ejemplo más sencillo posible \cite{Couairon2007} contempla exclusivamente cambios en la densidad debido a la fuerte ionización producida por el campo eléctrico durante su interacción con el medio, denominada \emph{Optical Field Ionization} (OFI). Durante el régimen llamado \emph{Multiphoton Absorption} (MPA), válido para $I \leq \qty{e13}{W/cm^2}$, varios fotones son absorbidos de forma simultánea por el medio y la densidad está descrita por \cite{Couairon2006b, Couairon2009}
\begin{equation}\label{eq:14}
  \pdv{\rho}{\tau} = \sigma_{K}\abs{\symcal{E}}^{2K}(\rho_{\mathrm{at}} - \rho),
\end{equation}
con $\sigma_{K}$ la \enquote{sección eficaz} para ionización multifotónica o \emph{Multiphoton Ionization} (MPI), correspondiente con la absorción de un número $K$ de fotones necesarios para superar el potencial de ionización de los átomos. 

La densidad de átomos inicial $\rho_{\mathrm{at}}$ es mucho mayor que la densidad de electrones libres con estas intensidades, así que normalmente $\rho_{\mathrm{at}} - \rho \approx \rho_{\mathrm{at}}$. Cuando las intensidades superan el umbral mencionado, la ionización debido al efecto túnel empieza a ser importante y formulaciones más generales para el mecanismo de ionización \cite{Keldysh1965, Perelomov1966, Mishima2002} son necesarias.

\subsection{Modelo completo}\label{sec:12}
Aunque el modelo sencillo permite reproducir algunas de las características básicas de la filamentación como, por ejemplo, \emph{pulse splitting} (en el sistema del pulso, formación de múltiples \enquote{subpulsos} que viajan en sentidos opuestos), las hipótesis son demasiado fuertes y poco realistas en muchas circunstancias. Para obtener mejores modelos, basta con mantener que la envolvente es lentamente variable (SVEA) en la dirección de propagación $z$, pero no necesariamente en el tiempo \cite{Couairon2006b}.

La introducción del tiempo es responsable del fenómeno de dispersión típico de cualquier onda electromagnética. Dentro del ámbito de los modelos numéricos \cite{Kolesik2004, Kolesik2001, Mlejnek1998, Mlejnek1999}, la dispersión por velocidad de grupo o \emph{Group Velocity Dispersion} (GVD) aparece siempre hasta segundo orden, mientras que modelos teóricos más generales \cite{Brabec2000, Couairon2006b} permiten duraciones de pulsos de varios ciclos ópticos y la generación de \emph{optical shocks} (que describen las desviaciones de la SVEA en el tiempo).

También hay que introducir todos los procesos despreciados durante la interacción entre los átomos, moléculas y electrones del medio. Por un lado, los electrones liberados por el láser pueden, en un proceso de \enquote{cascada} electrónica, acelerarse por bremsstrahlung inverso gracias al campo eléctrico del pulso e inducir nuevas ionizaciones. Por otra parte, los electrones pueden recombinarse \enquote{radiativamente} (transicionan de la banda de conducción a la de valencia) con sus respectivos huecos. En ambos casos, la evolución de la densidad electrónica vista en la ecuación \eqref{eq:14} es modificada por \cite{Mlejnek1998, Couairon2006b}
\begin{equation}\label{eq:15}
  \pdv{\rho}{\tau} = \sigma_{K}\abs{\symcal{E}}^{2K}(\rho_{\mathrm{at}} - \rho) + \frac{\sigma}{U_{i}} \rho \abs{\symcal{E}}^{2} - a \rho^{2},
\end{equation}
donde $\sigma$ es la sección eficaz de bremsstrahlung inverso y $U_{i}$ la energía de ionización de los átomos.

Respecto a la propagación de la envolvente, los cambios relacionados con los constituyentes del medio dan lugar a la ecuación final
\begin{equation}\label{eq:16}
  \begin{split}
    \pdv{\symcal{E}}{\zeta} & = i \frac{1}{2k} \laplacian_{\perp}\symcal{E} - i \frac{k^{\prime \prime}}{2}\pdvN{\symcal{E}}{t}{2} - \frac{\sigma}{2}(1 + i \omega \tau_{c}) \rho \symcal{E} - \frac{\beta^{(K)}}{2}\abs{\symcal{E}}^{2K-2}\symcal{E} \\
    & + i k_{0}n_{2}(1-\alpha)\abs{\symcal{E}}^{2}\symcal{E} + i k_{0}n_{2} \alpha \left(\int_{-\infty}^{t}R(t-t \prime)\abs{\symcal{E}(t \prime )}^{2} \diff t \prime\right)\symcal{E}.
  \end{split}
\end{equation}

Vamos a explicar todos los términos uno a uno. En primer lugar, el segundo término a la derecha de la igualdad es responsable de la dispersión, con $k \prime \prime$ el coeficiente de GVD de segundo orden comentado anteriormente. Este coeficiente tiene su origen en el desarrollo en serie de Taylor ---centrado en la frecuencia central $\omega$ del pulso--- del \enquote{operador} encargado de introducir los choques ópticos, es decir, $k \prime \prime = \partial^{2}k/\partial \omega^{2}$ y GVD es resultado de aproximar a orden dos el desarrollo.

En segundo lugar, el tercer término incluye absorción (parte real) y refracción (parte imaginaria) del pulso debido a la generación de electrones libres que forman el plasma. La constante $\tau_{c}$ es el tiempo medio de colisión entre electrones. En aire a \qty{800}{nm}, resulta que $\tau_{c} = \qty{350}{fs} \gg 1/\omega_{0} \approx \qty{0.4}{fs}$ \cite{Couairon2009} y la refracción puede volver a escribirse a partir de la densidad crítica del plasma $\rho_{c}$, porque entonces $\sigma \omega \tau_{c} \rho \approx k_{0} \rho/n_{0} \rho_{c}$, recuperando el término visto en la ecuación \eqref{eq:13}. 

El cuarto término procede del MPA ---como veremos en la sección \S\ref{sec:3}, participa en los cambios de intensidad del pulso---, con $\beta^{(K)} = K \hslash \omega \rho_{\mathrm{at}} \sigma_{K}$ la \enquote{sección eficaz} para el régimen de absorción multifotónica. El último término es el efecto Kerr óptico separada en una primera respuesta instantánea de las moléculas del medio, seguida de una segunda respuesta retardada. La segunda procede de la dispersión de las moléculas y, habitualmente \cite{Couairon2009,Couairon2006b,Couairon2007}, sigue una función de respuesta tipo Raman 
\begin{equation}\label{eq:17}
  R(t) = R_{0}\exp(-t/\tau_{R})\sin(\Lambda t),
\end{equation}
con $R_{0} = (1 + \Lambda^{2} \tau^{2}_{R})/\Lambda\tau^{2}_{R}$. A \qty{800}{nm} en aire, $\tau_{R} = \qty{70}{fs}$ y $\Lambda = \qty{16}{THz}$ representan el tiempo y frecuencia de respuesta de las moléculas, respectivamente.

\section{Soluciones en régimen lineal}\label{sec:2}
Cualquier forma de realizar comprobaciones y pruebas iniciales de soluciones aproximadas conviene, siempre que sea posible, compararlas con soluciones conocidas de las ecuaciones. La ecuación \eqref{eq:16} no tiene solución, pero eliminando todos los términos excepto la difracción y dispersión obtenemos
\begin{equation}\label{eq:18}
  \pdv{\symcal{E}}{z} = i \frac{1}{2k} \laplacian_{\perp} \symcal{E} - i \frac{k \prime \prime }{2}\pdvN{\symcal{E}}{t}{2},
\end{equation}
que resulta sí que tiene solución y puede construirse a partir del comportamiento de ambos términos separados.

La estrategia para buscar soluciones de las ecuaciones de difracción y dispersión son algo rebuscadas, pero son estándar y muy eficaces. Aquí vamos a seguir la presentación de \cite{Milonni2010} para haces Gaussianos (los apartados 7.5 y 7.6) y GVD (los apartados 8.4 y 8.5). Primero, la sección \S\ref{sec:21} presenta la solución para la difracción de un haz Gaussiano sin (y después añadiendo) sistema óptico que enfoque el haz. Segundo, la sección \S\ref{sec:22} se ocupa de la solución para GVD de un haz Gaussiano con un \emph{chirp} que modula la duración del pulso.

\subsection{Término de difracción}\label{sec:21}
El problema consiste en resolver la ecuación diferencial
\begin{equation}\label{eq:19}
  \pdv{\symcal{E}}{z} = i \frac{1}{2k} \left(\pdvN{\symcal{E}}{x}{2} + \pdvN{\symcal{E}}{y}{2}\right)
\end{equation}
para un haz Gaussiano inicial $\symcal{E}(x,y,z=0) = \symcal{E}_{0} \exp(-(x^{2}+y^{2})/w^{2}_{0})$ con $w_{0}$ la distancia desde el eje $x$ hasta el perfil del pulso, medida en dirección transversal.

Vamos a buscar soluciones $\symcal{E}(\symbf{r}) = \symcal{E}_{0}\exp(ik(x^{2}+y^{2})/2q(z))\exp(ip(z))$, donde las funciones $q$ y $p$ pueden tomar valores complejos. Nótese que si $q(z) = kw^{2}_{0}/2i$ y $p(z)$ toma valores reales, la intensidad del haz Gaussiano ---esto motiva un \enquote{truco} que viene después--- en un plano transversal a la dirección $\symbf{e}_{z}$ queda $I(x,y) = \symcal{E}^{2}_{0}\exp(-2(x^{2}+y^{2})/w^{2}_{0})$. Esta intensidad coincide con la medida en cualquier experimento para haces Gaussianos, así que la solución propuesta, en principio, tiene buena pinta.

Introduciendo la solución en la ecuación \eqref{eq:19} y reagrupando todos los términos, se obtiene
\begin{equation}\label{eq:22}
  k^{2}\frac{x^{2}+y^{2}}{q^{2}(z)}\left(\dv{q}{z} - 1\right) + 2k \left(\frac{i}{q(z)}-\dv{p}{z}\right) = 0.
\end{equation}

La ecuación solo puede satisfacerse para todo $x$, $y$, $z$ si ambos paréntesis se anulan. En ese caso, resulta que $q(z) = z + q_{0}$ y $p(z) = i\ln(1+z/q_{0})$ con $q_{0} = q(0)$ y $p(0) = 0$. En general, la elección de $q_{0}$ es arbitraria porque puedo escoger cualquier sistema de referencia, sin más que cambiar $q_{0}$ absorbiendo la constante resultante de $z \rightarrow z - d$. Ahora viene el truco comentado antes. 

Vamos a descomponer en parte real e imaginaria $q(z)$ tal que
\begin{equation}\label{eq:23}
  \frac{1}{q(z)} = \frac{1}{R(z)} + \frac{2i}{kw^{2}(z)},
\end{equation}
e imponemos que $R(0) \rightarrow \infty$ y $w(0) = w_{0}$ en un determinado sistema de referencia (la idea detrás consiste en suponer la existencia de un sistema de referencia donde $R(0) \rightarrow \infty$, y desplazar el origen del sistema original). Entonces, $q_{0} = kw^{2}_{0}/2i$ y llamando $z_{\mathrm{R}} = kw^{2}_{0}/2$ a la \enquote{longitud de Rayleigh}
\begin{equation}\label{eq:24}
  \frac{1}{z + q_{0}} = \frac{1/q_{0}}{1 + z/q_{0}} = \frac{z/z^{2}_{\mathrm{R}} + i/z^{2}_{\mathrm{R}}}{1 + (z/z_{\mathrm{R}})^{2}} = \frac{1}{R(z)} + \frac{2i}{kw^{2}(z)}.
\end{equation}

A partir de la ecuación \eqref{eq:24} se obtienen los conocidos por los nombres de \enquote{radio de curvatura} $R(z)$ y \enquote{ancho} $w(z)$ del haz Gaussiano, de manera que 
\begin{align}
  R(z) & = z + \frac{z^{2}_{\mathrm{R}}}{z}, \label{eq:25} \\
  w(z) & = w_{0}\sqrt{1 + (z/z_{\mathrm{R}})^{2}} \label{eq:26}.
\end{align}

Regresando a la función $p$ con esta información, puede comprobarse que
\begin{equation}\label{eq:27}
  \exp(ip(z)) = \frac{q_{0}}{z+q_{0}} = \frac{1}{1 + z/q_{0}} = \frac{1}{1 + iz/z_{\mathrm{R}}} = \frac{w_{0}}{w(z)}\exp(-i \Psi(z)),
\end{equation}
con $\phi(z) = \arctan(z/z_{\mathrm{R}})$ la llamada \enquote{fase de Gouy} del pulso. Recogiendo las ecuaciones \eqref{eq:25}, \eqref{eq:26} y \eqref{eq:27} e introduciéndolas en la solución general buscada obtenemos la expresión para la envolvente de un pulso Gaussiano 
\begin{equation}\label{eq:28}
  \symcal{E}(\symbf{r}) = \symcal{E}_{0}\frac{w_{0}}{w(z)}\exp\left(-\frac{x^{2}+y^{2}}{w^{2}(z)} + \frac{ik(x^{2}+y^{2})}{2R(z)} - i \phi(z)\right).
\end{equation}
 
A partir de la solución $\symcal{E}$ conseguida, está claro que cualquier haz Gaussiano inicial mantiene su forma durante la propagación en el vacío. Los radios de curvatura del haz tienden a cero cuando nos acercamos al origen del sistema de referencia, recuperando la condición inicial esperada. A medida que avanza el pulso, los frentes de onda aumentan su curvatura y dejan de ser planos.

Una vez encontrada la solución para el espacio vacío, el siguiente paso consiste en encontrar como se ve afectada la propagación del haz Gaussiano después de interponer una lente de distancia focal $f$. En especial, hay que determinar la transformación general de la función $q$ debido a un sistema óptico como una lente, para después viajar una determinada distancia $d$ en el vacío. Colocando la lente en el origen del sistema de referencia ---en la posición con anchura $w(0) = w_{0}$ mínima---, la función $q$ inicialmente era $q_{i} = kw^{2}_{0}/2i = -i \pi w^{2}_{0}/\lambda$.

Esta colocación estratégica de la lente justo donde el pulso tiene su anchura mínima permite, recuperar la condición inicial utilizada para la envolvente, pero introduciendo un factor de fase ---a priori--- desconocido debido al enfoque de la lente. Falta averiguar como cambian las funciones que aparecen en la ecuación \eqref{eq:28} a partir de la transformación del parámetro $q_{i}$ inicial. 

La ley \emph{ABCD} para una lente delgada de distancia focal $f$ seguida de un desplazamiento hasta una distancia $z=d$ es
\begin{equation}\label{eq:29}
  \begin{pmatrix}
   A & B \\
   C & D
  \end{pmatrix} 
  =
  \begin{pmatrix}
   1 - d/f & d \\
    -1/f & 1
  \end{pmatrix},
\end{equation}
de manera que el parámetro $q_{f}$ resultante de transformar $q_{i}$ está relacionado con la aplicación lineal dada por esta matriz
\begin{equation}\label{eq:30}
  q_{f} = \frac{Aq_{i}+B}{Cq_{i}+D} = \frac{-(i \pi w^{2}_{0}/\lambda)(1 - d/f) + d}{i \pi w^{2}_{0}/\lambda f + 1} = \left(\frac{1}{R(d)} + \frac{i \lambda}{\pi w^{2}(d)}\right)^{-1}.
\end{equation}
Volviendo a igualar la parte real e imaginaria en la ecuación \eqref{eq:30} y agrupando términos obtenemos las expresiones
\begin{align}
  w^{2}(d) & = w^{2}_{0}\left(1 - \frac{d}{f}\right)^{2} + w^{2}_{0}\left(\frac{d}{z_{\mathrm{R}}}\right)^{2}, \label{eq:31} \\
  R(d) & = \frac{(d/z_{\mathrm{R}})^{2} + (1-d/f)^{2}}{d/z^{2}_{\mathrm{R}} - (1/f)(1-d/f)}, \label{eq:32}
\end{align}
para el radio de curvatura $R(d)$ y el ancho transversal $w(d)$ del pulso a cualquiera distancia $d$ detrás de la lente. La anchura mínima del pulso ---donde encontraríamos el \enquote{nuevo} $w_{0}$, después de enfocarse--- está en la coordenada donde el frente de onda es plano, es decir, $R(d) \rightarrow \infty$. Por ejemplo, derivando el radio de curvatura en la ecuación \eqref{eq:31} para encontrar el mínimo, vemos que
\begin{equation}\label{eq:33}
  \frac{1}{f}\left(1-\frac{d}{f}\right) - \frac{d}{z^{2}_{\mathrm{R}}} = 0,
\end{equation}
luego $d = f/(1 + f^{2}/z^{2}_{\mathrm{R}})$ es la distancia detrás de la lente donde el nuevo ancho $w_{0}$ está localizado.

Sustituyendo $d$ en la ecuación \eqref{eq:31} del ancho mínimo se obtiene
\begin{equation}\label{eq:34}
  w_{f} = w_{0}\frac{f}{\sqrt{z^{2}_{\mathrm{R}} + f^{2}}},
\end{equation}
detrás de la lente a una distancia $d$ y, de forma totalmente análoga, podemos definir su longitud de Rayleigh correspondiente como $z_{f} = k_{0}w^{2}_{f}/2$.

Haciendo un poco de álgebra, es fácil comprobar que $z^{2}_{f} = d(f-d)$, así que el radio de curvatura $R(z)$ y la fase de Gouy $\phi(z)$ detrás de la lente pueden conocerse a partir de las ecuaciones \eqref{eq:25} y \eqref{eq:26}, trasladando ambas funciones hasta la nueva posición del ancho mínimo $z \rightarrow z - d$ y cambiando la longitud de Rayleigh $z_{\mathrm{R}} \rightarrow z_{f}$. De esta forma, la envolvente mantiene la forma dada por la anterior ecuación \eqref{eq:28}, pero ahora con
\begin{align}
  w(z) & = w_{0}\sqrt{(1 - z/f)^{2} + (z/z_{\mathrm{R}})^{2}}, \label{eq:35} \\
  R(z) & = z - d + \frac{z^{2}_{f}}{z-d}, \label{eq:36} \\
  \phi(z) & = \arctan \left(\frac{z-d}{z_{f}}\right).
\end{align}

\subsection{Término de dispersión por velocidad de grupo}\label{sec:22}
Después de encontrar la solución general para el problema de la difracción, hay que resolver la ecuación de dspersión
\begin{equation}\label{eq:37}
  \pdv{\symcal{E}}{z} = -i \frac{k \prime \prime }{2} \pdvN{\symcal{E}}{t}{2}
\end{equation}
con un perfil temporal inicial dado por
\begin{equation}\label{eq:38}
  \symcal{E}(z=0,t) = \symcal{E}_{0} \exp\left(-(1+iC)t^{2}/t^{2}_{\mathrm{p}}\right),
\end{equation}
siendo el parámetro $C$ un \emph{chirp} encargado de modificar la duración del pulso y $t^{2}_{\mathrm{p}}$ el tiempo de pico. El objetivo detrás de este perfil de fase dependiente del tiempo es que, en un medio con dispersión \enquote{normal} $k \prime \prime > 0$ y $C < 0$, la duración del pulso puede comprimirse al comienzo de la propagación. En cambio, con $C > 0$ la duración del pulso crece indefinidamente durante su viaje.

El planteamiento es equivalente ---cambiando $z \rightarrow t$ y $t \rightarrow x$--- a resolver la ecuación de Schrödinger para una partícula libre en movimiento (porque el paquete de ondas inicial no es estacionario, sino viajero). La transformada de Fourier es la forma más rápida de abordar el problema de encontrar la solución, igual que en mecánica cuántica cuando buscan soluciones en el espacio de momentos. Aplicando la transformada de Fourier (en la frecuencia angular $\omega$) a la ecuación \eqref{eq:37}
\begin{equation}\label{eq:39}
  \pdv{\symcal{E}_{\mathrm{F}}}{z} = -i \frac{k \prime \prime}{2} \omega^{2} \symcal{E}_{\mathrm{F}},
\end{equation}
llamando $\symcal{E}_{\mathrm{F}}$ a la transformada de Fourier de la envolvente $\symcal{E}$. Integrando la ecuación, la solución en el espacio de frecuencias es $\symcal{E}_{\mathrm{F}}(z,\omega) = \symcal{E}_{\mathrm{F}}(z=0,\omega)\exp(ik \prime \prime \omega^{2}z/2)$.

Ahora es necesario calcular dos integrales, una para transformar la condición inicial y otra para deshacer la transformación. En los dos casos, hay que completar cuadrados en funciones exponenciales de la forma $\exp(-at^{2} \pm i \omega t)$ para integrar una función Gaussiana. La primera integración es
\begin{equation}\label{eq:40}
  \symcal{E}_{\mathrm{F}}(z=0, \omega) = \symcal{E}_{0} \int_{-\infty}^{\infty} \exp(-at^{2} - i \omega t) \diff t = \symcal{E}_{0}\sqrt{\frac{\pi}{a}}\exp\left(-\frac{\omega^{2}}{4a}\right)
\end{equation}
con $a = (1+iC)/t^{2}_{\mathrm{p}}$. La solución de la ecuación \eqref{eq:39} es entonces
\begin{equation}\label{eq:41}
  \symcal{E}_{F}(z,\omega) = \symcal{E}_{0}\sqrt{\frac{\pi}{a}}\exp\left(-\frac{\omega^{2}}{4a} + i \frac{k \prime \prime}{2} \omega^{2} z\right) = \symcal{E}_{0}\sqrt{\frac{\pi}{a}}\exp(-b \omega^{2}),
\end{equation}
para $b = 1/4a - ik \prime \prime z/2$. Aplicando la transformada inversa de Fourier a la ecuación \eqref{eq:41} encontramos la solución final
\begin{equation}\label{eq:42}
  \begin{split}
    \symcal{E}(z,t) & = \frac{1}{2 \pi} \symcal{E}_{0}\sqrt{\frac{\pi}{a}}\int_{-\infty}^{\infty}\exp(-b \omega^{2} + i \omega t) \diff \omega = \frac{1}{2 \pi} \symcal{E}_{0}\sqrt{\frac{\pi}{a}}\sqrt{\frac{\pi}{b}}\exp \left(-\frac{t^{2}}{4b}\right) \\
    & = \symcal{E}_{0}\frac{1}{\sqrt{4ab}} \exp \left(-\frac{t^{2}}{4b}\right).
  \end{split}
\end{equation}

Después de mucho álgebra, puede demostrarse que la solución final tiene la forma
\begin{equation}\label{eq:43}
  \symcal{E}(z,t) = \symcal{E}_{0}\sqrt{\frac{t_{\mathrm{p}}}{T(z)}}\exp \left(-\frac{t^{2}}{T^{2}(z)}\left(1+i\left(C+(1+C^{2})\frac{z}{z_{\mathrm{ds}}}\right)\right) - i \psi(z)\right),
\end{equation}
siendo
\begin{align}
  T(z) = t_{\mathrm{p}}\sqrt{(1+Cz/z_{\mathrm{ds}})^{2} + (z/z_{\mathrm{ds}})^{2}}, \label{eq:44} \\
  \psi(z) = \frac{1}{2}\arctan \left(-\frac{z}{Cz + z_{\mathrm{ds}}}\right), \label{eq:45}
\end{align}
la duración de pulso y la fase, respectivamente. El análogo de la longitud de Rayleigh para la difracción es la \enquote{longitud de dispersión}, definida como $z_{\mathrm{ds}} = t^{2}_{\mathrm{p}}/2k \prime \prime$ para nuestro haz inicial. 

En \cite{Couairon2011} (ver sección 3.1.3) hay tres erratas que pueden inducir a confusión. Para empezar, la amplitud de la envolvente depende la raíz cuadrada de $t_{\mathrm{p}}/T(z)$, así que cualquier programa bien escrito para calcular soluciones aproximadas representará una curva distinta. La fase también es incorrecta porque el argumento de $\arctan$ tiene dimensiones, mientras que la fase en la ecuación \eqref{eq:45} es adimensional como corresponde (aunque a efectos prácticos, este factor de fase no afecta a la intensidad de pulso). 

En el párrafo que va inmediatamente después, aseguran que la duración mínima del pulso toma el valor $T_{\mathrm{m}} = t_{\mathrm{p}}/(1+C^{2})$, pero puede comprobarse fácilmente derivando que en realidad es $T_{\mathrm{m}} = t_{\mathrm{p}}/\sqrt{1+C^{2}}$ (otra raíz cuadrada que falta). La solución numérica debería entonces sobreestimar la duración del pulso, como así sucede cuando comparas la solución real con la aproximada. En cambio, la posición del mínimo $z_{\mathrm{m}} = -Cz_{\mathrm{ds}}/(1+C^{2})$ sí que está bien calculada.

La solución completa de la ecuación \eqref{eq:18} puede construirse a partir de las soluciones para la disfracción y dispersión por velocidad de grupo multiplicándolas, porque la ecuación diferencial es lineal en todos sus términos. En resumen, si la solución del primer término se escribe como $Af(x,y,z)$ y la del segundo $Ag(z,t)$, la solución de la ecuación completa es $Af(x,y,z)g(z,t)$.

\section{Acumulación de fase e intensidad}\label{sec:3}
Conocer en profundidad la evolución de la intensidad del pulso es importante porque ayuda a comprobar si las soluciones obtenidas numéricamente tienen errores. También permite reconocer el papel que juegan los distintos fenómenos físicos que participan en la filamentación y su relación con la envolvente inicial que comienza la propagación.

\subsection{Gradientes espaciales de fase}\label{sec:31}
Vamos a fijarnos primero en la ecuación \eqref{eq:19} para la difracción. La derivada del módulo al cuadrado de la envolvente es
\begin{equation}\label{eq:46}
  \pdv{|\symcal{E}|^{2}}{z} = \pdv{\symcal{E}^{*}}{z}\symcal{E} + \pdv{\symcal{E}}{z}\symcal{E}^{*} = i \frac{1}{2k}(\symcal{E}^{*} \laplacian_{\perp}\symcal{E} - \symcal{E} \laplacian_{\perp}\symcal{E}^{*}).
\end{equation}

La envolvente en cualquier punto del espacio y del tiempo es un número complejo, así que podemos escribirla en forma módulo-argumento como $\symcal{E}=|\symcal{E}|\exp(i \theta)$. El término entre paréntesis puede reducirse dándose cuenta de que
\begin{align}
  \symcal{E}^{*}\laplacian_{\perp}\symcal{E} & = \grad_{\perp} \cdot (\symcal{E}^{*}\grad_{\perp}\symcal{E}\,) - (\grad_{\perp}\symcal{E}^{*}) \cdot (\grad_{\perp}\symcal{E}\,), \\
  \symcal{E}\laplacian_{\perp}\symcal{E}^{*} & = \grad_{\perp} \cdot (\symcal{E}\,\grad_{\perp}\symcal{E}^{*}) - (\grad_{\perp}\symcal{E}\,) \cdot (\grad_{\perp}\symcal{E}^{*}),
\end{align}
quedando la ecuación \eqref{eq:46} como
\begin{equation}\label{eq:47}
  \pdv{|\symcal{E}|^{2}}{z} = i \frac{1}{2k}\grad_{\perp}\cdot(\symcal{E}^{*} \grad_{\perp}\symcal{E} - \symcal{E}\,\grad_{\perp}\symcal{E}^{*})
\end{equation}
y
\begin{equation}\label{eq:47}
  \symcal{E}^{*}\grad_{\perp}\symcal{E} = |\symcal{E}|\exp(-i \theta)(i|\symcal{E}|\exp(i \theta)\grad_{\perp} \theta + \exp(i \theta)\grad_{\perp}\symcal{E}) = i|\symcal{E}|^{2}\grad_{\perp} \theta + \symcal{E}\,\grad_{\perp}\symcal{E}.
\end{equation}

A partir de la ecuación \eqref{eq:47}, el término entre paréntesis queda $(\symcal{E}^{*} \grad_{\perp}\symcal{E} - \symcal{E}\,\grad_{\perp}\symcal{E}^{*}) = 2i|\symcal{E}|^{2}\grad_{\perp} \theta$ y, por tanto, sustituyendo en la ecuación \eqref{eq:46} resulta que
\begin{equation}\label{eq:48}
  \pdv{|\symcal{E}|^{2}}{z} = - \frac{1}{k}\grad_{\perp} \cdot (|\symcal{E}|^{2}\grad_{\perp} \theta).
\end{equation}

La conclusión que puede extraerse de esta igualdad es que los gradientes de fase del pulso son responsables de los cambios de intensidad del pulso en la dirección de propagación. El gradiente espacial de la fase produce un flujo de energía en dirección radial hacia el centro del haz, incrementando la intensidad en su región central (donde está el filamento propiamente dicho). Además, es sencillo ver que añadir términos como el efecto Kerr o la refracción del pulso debido al plasma de electrones, no repercuten en absoluto sobre la evolución de la intensidad.

En realidad, el propósito del efecto Kerr y la refracción del plasma es producir una \enquote{acumulación} de fase para, posteriormente, ser mediada por otros fenómenos encargados de generar un rápido incremento de intensidad del pulso. La difracción es uno de estos procesos y, como se explica en la sección \S\ref{sec:32}, la dispersión juega un papel similar.

Los fenómenos de absorción debido al plasma, como era de esperar, también repercuten en la evolución de intensidad. Por ejemplo, puede comprobarse que la absorción por MPA disminuye la intensidad del pulso según $\partial \symcal{E}/\partial \zeta = -\beta^{(K)}|\symcal{E}|^{2K}$, que es siempre negativa.

\subsection{Variación temporal de fase}\label{sec:32}
Debido a la dispersión del pulso, los cambios de fase con el tiempo también son importantes en la evolución de la intensidad. La ecuación \eqref{eq:37} produce una variación de la intensidad dada por
\begin{equation}\label{eq:48}
  \pdv{|\symcal{E}|^{2}}{z} = i \frac{k^{(2)}_{0}}{2} \left(\symcal{E}\pdvN{\symcal{E}^{*}}{t}{2} - \symcal{E}^{*}\pdvN{\symcal{E}}{t}{2}\right).
\end{equation}

Los pasos a seguir para encontrar la relación entre acumulación de fase e intensidad son iguales a la sección \S\ref{sec:31}, sustituyendo el operador $\grad_{\perp}$ por derivadas respecto al tiempo y el factor $i/2k$ por $-ik \prime \prime/2$. Entonces, la intensidad cambia según la ecuación
\begin{equation}\label{eq:49}
  \pdv{|\symcal{E}|^{2}}{z} = k \prime \prime \frac{\partial}{\partial t} \left(|\symcal{E}|^{2}\pdv{\theta}{t}\right).
\end{equation}

Una apreciación importante, especialmente durante la realización de simulaciones numéricas, es que en régimen de propagación lineal es necesario introducir un factor de fase para el pulso Gaussiano inicial. El factor de fase inicial tiene que depender de las coordenadas transversales o del tiempo, porque en caso contrario no habrá evolución de la intensidad. Por el contrario, cuando incluyes términos adicionales como absorción o MPA para estudiar el modelo completo, desaparece ese requisito.

\section{Métodos numéricos}\label{sec:4}
Las técnicas de resolución de las ecuaciones de propagación están muy bien explicadas (entre las secciones 3.1 y 3.3) en \cite{Couairon2011}. A modo resumen, hay dos maneras distintas de simular el modelo completo presentado en la sección \S\ref{sec:12}. El método preferido por los autores consiste en resolver mediante la transformada rápida de Fourier el término de dispersión en un primer paso y utilizar un esquema de Crank-Nicolson \enquote{extendido} para los términos de difracción y demás procesos físicos durante el segundo paso.

Este primer método, que suele llamarse \emph{Crank-Nicolson Fourier} (FCN) \cite{Couairon2007}, es un método espectral bastante rápido y muy eficaz. El concepto de Crank-Nicolson extendido hace referencia a que la difracción está siendo resuelta por Crank-Nicolson, mientras que los demás términos utilizan algún esquema distinto, normalmente métodos multipaso tipo Adam-Bashforth o métodos de Runge-Kutta. El Adam-Bashforth de dos pasos (el que están sugiriendo) conserva la precisión de orden dos del Crank-Nicolson y es sencillo de implementar, mientras que Runge-Kutta u otros métodos implícitos aumentan mucho el coste computacional.

La ecuación de evolución de la densidad electrónica puede resolverse (ver sección 3.3.1 de \cite{Couairon2011}) integrando numéricamente en cada paso temporal (opción sencilla) o utilizando también un esquema de Runge-Kutta (mencionado en la sección 2.3 de \cite{Couairon2007}, más complejo y con más coste computacional). Este primer método FCN también puede hacerse completamente en el espacio espectral de Fourier, en lugar de separando los operadores de dispersión y difracción en dos pasos.

El segundo método (mencionado en la sección 2.3 de \cite{Couairon2007}) utiliza el método de direcciones alternantes entre la coordenada radial y temporal, conocido como \emph{Alternate Direction Implicit} (ADI). Este sistema permite resolver los operadores de difracción y dispersión en dos semipasos, pero en cada uno de ellos una dimensión se resuelve implícitamente y otra explícitamente. Durante el siguiente semipaso, simplemente invierte los papeles de ambos operadores, resolviendo implícitamente la dirección antes resuelta explícitamente y viceversa (ver la sección 3.1.3 de \cite{Couairon2011}). Los sistemas de ecuaciones a resolver en el bucle principal son
\begin{align}
  L_{\delta_{-}}\symbf{U}^{k + 1/2} & = L_{\epsilon_{+}}\symbf{U}^{k}, \label{eq:50} \\
  L_{\eta_{-}}\symbf{U}^{k + 1} & = L_{\delta_{+}}\symbf{U}^{k + 1/2}, \label{eq:51}
\end{align}
con $k$ el índice asociado a la discretización de la coordenada $z$ de propagación. Los símbolos $L_{\delta_{\pm}}$ y $L_{\eta_{\pm}}$ son las matrices de Crank-Nicolson para la difracción y la dispersión, respectivamente. El símbolo $\symbf{U}$ es la matriz de la envolvente que almacena los valores para las coordenadas radial y temporal.

En cada paso de $k$ hay que realizar cuatro operaciones matriciales: dos productos matriciales (lado derecho de las ecuaciones) para encontrar un vector $\symbf{b}$, y dos soluciones de sistemas matriciales tipo $A \symbf{U} = \symbf{b}$ (lado izquierdo de las ecuaciones). Como las matrices de Crank-Nicolson son dispersas (porque son tridiagonales), incluso almacenando las matrices como \emph{sparse} el número de operaciones es alto.

Para dar una idea de la potencia de ambos métodos, en régimen lineal el método FCN consigue reproducir con gran precisión (el error es $\sim 0,01\%$) la solución en aproximadamente \qty{480}{s} (8 minutos), mientras que ADI consigue el mismo resultado en \qty{1780}{s} (28 minutos). Aunque pueda parecer más razonable utilizar FCN, es cierto que una exploración más detallada del algoritmo (en estos casos Python, pero podría ser cualquier lenguaje) podría reducir el número de operaciones realizadas en cada paso del bucle. 

Durante estas últimas pruebas, las comparaciones entre ambos sistemas se han realizado con éxito utilizando $1002$ nodos en la coordenada radial (dos nodos \enquote{fantasma} para imponer condiciones de frontera tipo Dirichlet (homogéneas)-Neumann), $1002$ nodos en la coordenada temporal para el método ADI (otros dos nodos fantasma para imponer condiciones tipo Dirichlet homogéneas) o $1000$ nodos para el método FCN (las condiciones son periódicas e impuestas automáticamente por el algoritmo de FFT) y $1000$ iteraciones en la coordenada longitudinal de propagación.

%%% bibliografía
\printbibliography

\end{document}
